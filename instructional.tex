\documentclass{article}

\title{Visualisation, harnessing the eye of the mind.}
\date{2020-19-12}
\author{0x4F}

\begin{document}
\pagenumbering{gobble}
\maketitle
\newpage

\section{What is visualisation?}
What I call ``Visualisation'' is a practice based on the Gateway Process to induce a state of deep relaxation. The effect of this state is that your body falls asleep, while your mind remains conscious. During the visualising state, the visualiser's minds eye will be brought to the forefront, in what are called Hypnagogic Hallucinations. Once this point is reached, the visualiser has essentially hacked in to a direct connection with their subconscious. The visualiser has ultimate control over the Hypnagogic Dream World and can shape it and reshape it at their will. The visualiser can also allow the subconscious to shape the Dream World, and that often leads to very deep personal insights and experiences. Within the Dream World there can also be NVC's (non visualiser characters) which can either be controlled and programmed by the visualiser, or by the subconscious. Visualisation is both a relaxing, fun, and enlightening experience; and if you decide to try it, you won't be dissapointed.

There are rare cases, that can easily be avoided, where a visualisers dream world can be entered by intelligent non-corporeal autonomous entities. In most cases these entities are helpful, and can often provide insight for the visualiser, but there are also cases where the entity can attempt to hijack the visualisation to induce a nightmare-like Dream World. How to deal with these entities will be explained with in another section.


\section{Expectations and experiences}

\section{Common experiences}
There a few common details shared among the people who have researched Visualisation with me. One of the first is the effect that gives the practice it's name; the visuals. The visuals of the Visualisation is largely up to the creativity and level of skill in shaping/creating mental imagery of the one visualising. Like most skills this can take a bit of practice, but getting good at it is well worth it. An easy place to start is imagining yourself walking on a placid ocean at night. The second common experience is physical; often you'll know that you're entering the Visualisation state by feeling one or more of the following sensations, vertigo, loss of feeling in body, tingling in hands, and a feeling of pressure on the chest. The third common experience is Subconscious bleeding; this occurs when the visualiser is not actively controlling the Dream World and the subconscious bleeds in. Often this effect takes the form of people, objects, or locations appearing in the Dream World. For example, I had placed myself within a cafe situated in a surreal cityscape, and found myself joined by a friend that I had been worried about at the time.

\subsection{}

\subsection{Personal experiences}

\subsection{A Simple recipe}

\section{Visualisation preparation and practice}

\subsection{Pre-requisites}
Dual channel stereo headphones. These are required due to the way the binaural beats effect works. Quality of headphones is of marginal importance.
A room to visualise that should ideally be a place you feel calm and safe.
Optionally, a sleep or eye mask. This is useful if you cannot get your visualisation room dark.


\subsection{Clearing your mind Pre-visualisation}
Often distracting thoughts, and situations our mind is preoccupied with can and will interrupt passage to the   necessary states of relaxation required to take all benefits from Visualisation.
This kind of distraction is very common, but also quite easily mitigated. To mitigate and clear your mind before  attempting a session of Visualisation, allocate ten minutes to listening to music of your
choice with your eyes closed. During this time, allow any thoughts you wish to think to make themselves clear, don't hold yourself back from whatever you need to think about. If after ten minutes of
this you still have thoughts that require attention, try another ten minutes. You'll know that you are ready when your mind is feeling clearer, and you are feeling calmer.


\subsection{Preparation}
For an optimal experience, the visualiser should remove as many extraneous distractions as possible. Lights, electronics, and background noise, all hinder and mitigate the effects of the tapes. Thus participants should attempt to avoid to the best of their abilities the distractions that follow:
\begin{itemize}
  \item Light of any kind.
  \item Electronics.
  \item Noise.
  \item Heavy or recurring thoughts.
  \item Feelings of anxiety or worry.
  \item Pets or animals.
  \item Aches and pain.
\end{itemize}
It is best to also avoid caffeine and other stimulants before engaging visualisation as they have a stressing effect upon the body and may impede the relaxation required for the full experience.
For other preparation, before beginning the visualisation it is advised to clear your mind, and relax as fully as possible.
The final step of preparation is placing yourself in a comfortable position where you will feel as little of your body as possible.

\subsection{Tapes}
You should ideally begin with wave 1, tape 1, and continue from there in a linear order. If you make it through wave 1 and don't feel like the experience has been particularly effective, it's
recommended to go through the entirety of wave 1 multiple times, until you feel ready to move on.

\section{Closing notes}
This is the end of the document. As a final point, I'd like to include some of the currently experienced and confirmed effects of the tapes. The most powerful effect of the tapes is the relaxed state it puts the listener in. The second, less powerful, but more notable effect of the tapes is channeled through the minds eye. If you have an aptitude for visualisation, it can be expected that you will be able to craft incredibly vivid visualisations, and then perceive them within your closed eyes. These visualisations share a similar quality to dreams, but offer far more control, and creativity. For examples of visualisation, see the logs.

\end{document}
